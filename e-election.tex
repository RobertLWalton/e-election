% Electronic Election System
%
% File:         e-election.tex
% Author:       Bob Walton (walton@acm.org)
% Date:		See \date below.

\documentclass[12pt]{article}

\usepackage{times}
\usepackage{makeidx}

\makeindex

\setlength{\oddsidemargin}{0in}
\setlength{\evensidemargin}{0in}
\setlength{\textwidth}{6.5in}
\raggedbottom

\setlength{\unitlength}{1in}

\pagestyle{headings}
\setlength{\parindent}{0.0in}
\setlength{\parskip}{1ex}

% Begin \tableofcontents surgery.

\newcount\AtCatcode
\AtCatcode=\catcode`@
\catcode `@=11	% @ is now a letter

\renewcommand{\contentsname}{}
\renewcommand\l@section{\@dottedtocline{1}{0.1em}{1.4em}}
\renewcommand\l@table{\@dottedtocline{1}{0.1em}{1.4em}}
\renewcommand\tableofcontents{%
    \begin{list}{}%
	     {\setlength{\itemsep}{0in}%
	      \setlength{\topsep}{0in}%
	      \setlength{\parsep}{1ex}%
	      \setlength{\labelwidth}{0in}%
	      \setlength{\baselineskip}{1.5ex}%
	      \setlength{\leftmargin}{0.8in}%
	      \setlength{\rightmargin}{0.8in}}%
    \item\@starttoc{toc}%
    \end{list}}
\renewcommand\listoftables{%
    \begin{list}{}%
	     {\setlength{\itemsep}{0in}%
	      \setlength{\topsep}{0in}%
	      \setlength{\parsep}{1ex}%
	      \setlength{\labelwidth}{0in}%
	      \setlength{\baselineskip}{1.5ex}%
	      \setlength{\leftmargin}{1.0in}%
	      \setlength{\rightmargin}{1.0in}%
	      }%
    \item\@starttoc{lot}%
    \end{list}}

\catcode `@=\AtCatcode	% @ is now restored

% End \tableofcontents surgery.

\newcommand{\CN}[2]%	Change Notice.
    {\hspace*{0in}\marginpar{\sloppy \raggedright \it \footnotesize
     $^{\mbox{#1}}$#2}}
    % Change notice.

\newcommand{\figref}[1]{\ref{#1}$\,^{p\,\pageref{#1}}$}
\newcommand{\pagref}[1]{(see page~\pageref{#1} below)}
\newcommand{\pagnote}[1]{$\,^{p\,\pageref{#1}}$}

\newcommand{\EOL}{\penalty \exhyphenpenalty}

\newlength{\figurewidth}
\setlength{\figurewidth}{\textwidth}
\addtolength{\figurewidth}{-0.40in}

\newsavebox{\figurebox}

\newenvironment{boxedfigure}[1][!btp]%
	{\begin{figure*}[#1]
	 \begin{lrbox}{\figurebox}
	 \begin{minipage}{\figurewidth}

	 \vspace*{1ex}}%
	{
	 \vspace*{1ex}

	 \end{minipage}
	 \end{lrbox}
	 \begin{center}
	 \fbox{\hspace*{0.1in}\usebox{\figurebox}\hspace*{0.1in}}
	 \end{center}
	 \end{figure*}}

\newenvironment{indpar}[1][0.3in]%
	{\begin{list}{}%
		     {\setlength{\itemsep}{0in}%
		      \setlength{\topsep}{0in}%
		      \setlength{\parsep}{1ex}%
		      \setlength{\labelwidth}{#1}%
		      \setlength{\leftmargin}{#1}%
		      \addtolength{\leftmargin}{\labelsep}}%
	 \item}%
	{\end{list}}


\begin{document}
        
\title{An Electronic Election System}

\author{Robert L. Walton\thanks{Copyright 2019 Robert L. Walton.
Permission to copy this document {\bf verbatim} is granted by the author
to the public.}}

\date{July 23, 2019}

\maketitle

\begin{center}
\large \bf Table of Contents
\end{center}

\bigskip

\tableofcontents 

\newpage

\section{Introduction}

Our purpose is to specify an all-electronic election system
suitable for replacing the current optical paper system
used in Massachusetts.  Our system is based on the notion
of `tamper proof software', which is software that is
known to do what it should do.

\section{Outline}

A Private Voting Machine
(PVM) is constructed from a single thumb drive and a
registration card.  Part of the thumb drive contains the
tamper proof software of the PVM, part contains an encryped
data set called a Voter Lock Box (V-Box), and the registration
card contains a passkey that is entered on the keyboard in
order to decrypt/encrypt the V-Box.  The software runs on
a (fairly) arbitrary computer, is bootstrapped directly,
runs without using any operating system or BIOS,
uses only RAM memory during operation,
writes the encryped V-Box back to the thumb drive, and sends
a very few encrypted messages to other `machines' in the
system.

The other machines are the Registration Machines, which make
ready-to-mark ballots available and time-stamp and certify
encrypted finished ballots; the Collection Machines that collect
finished ballots; and the Counting Machines that count the
ballots.  Each machine has the same design as the PVM, except
these other machines are each run by an election official, and
each are replicated with different officials running different
replicas.  Each Counting Machine also uses separate
thumb drives containing master lock boxes
(M-Boxes), and requires 5 out of 10 election officials to cooperatively
provide decrypted M-Boxes in order to decrypt ballots and
produce final counts.

Each Machine and each M-Box as a public key and a private key.
Each precinct has its own separate set of Counting Machines and M-Boxes with
private keys, but precincts may share Collection Machines and Registration
Machines.

Collection Machines, Counting Machines, and M-Boxes may be
specific to a particular election, but Private Voting Machines and
Registration Machine URLs and public keys are shared across
multiple elections.

\section{Tamper Proof Software}

Tamper Proof Software is software written in a `tamper proof'
programming language
that has the following property:
\begin{indpar}
The translation of souce code to binary files is so precisely
specified that several independent groups of programmers can
write compilers that will produce identical binaries from any
given source.
\end{indpar}

Clearly a sufficiently simple assembly language qualifies, but
we need a somewhat higher level language.  Still, such a language
should be possible.  Two aids to this end are:
\begin{enumerate}
\item
Computer efficiency is not required in this application.

\item
If two compilers disagree on some source, the matter can be
investigated, and either one or both compilers fixed, or the
language specification improved.
\end{enumerate}

The real goal is obtaining binaries that can be trusted to
do what the source code says they should do.  The other
property of tamper proof software is that the source code
is published, so the world can see what it does and look
for flaws.

The simplest tamper proof programming language is a straight
forward simple assembly language.  Given this, one can write
an interpreter for a simple interpreted language in the
assembly language.  The interpreted language should suffice
for most of the code, and can be machine independent.
A few functions can be optimized by being written in assembly
language.

\section{Keys and Names}

Each locked box in the system has a passkey, which is
written on a registration card associated with the locked
box.  Passkeys are generated by the system, and not by
people.  Typical passkeys consist of 2 or 3 pseudo-words
and 4 to 6 digits, chosen at random.  An example is
`vox64bam45zip01'.  A pseudo-word is equivalent to 3
digits, so this example is the equivalent of 15 random
digits, but the pseudo-words if chosen to be pronouncable
tend to make the passkey easier to type.

Each machine has a locked box.  Each Counting
Machine has 10 additional locked M-Boxes.

Each machine and each M-Box has a public key and a private key.  These
are long, e.g. 4,096 bit, values constructed randomly
in public/private pairs.

Each Registration and Collection
machine has a URL which serves as its name.

Each Private Voting Machine and each Counting Machine
has a Unique ID, or UID, which is a 1024 bit random number
used to identify the machine.  This UID serves as the
machine's name.

The correspondence between Private Voting Machine UIDs
and voter names and addresses is not part of the E-Election system
described here.  The E-Election system only knows about
Private Voting Machines identified by their UIDs.
Generally this correspondence would only be made available
to organizations and people authorized by state law.
For example, Massachusetts currently makes voter registration lists
available to registered political parties.

When one machine sends a message to another, the sender
encrypts the message proper with the sender's private
key, and attaches the unencrypted sender's name to the
message.  The receiver is expected to have the public
key of the sender.  Public keys are not published,
though the system is designed so that publishing public
keys will not compromise security or essentual privacy,
and voters can find out the public keys of Registration
Machines and Collection Machines by examining their
decrypted PVM Lock Box using deviant software
\pagref{DEVIANT-SOFTWARE}.

\section{Registration Machine Details}

A Registration Machine is run by a single election official
who holds the registration card containing the the passkey to
the machine's locked box.  The Registration Machine in this
sense is specific to a particular election, but the
URL naming the Registration Machine and the machine's
public/private key pair will typically remain the same from
one election to the next.

The locked box of a Registration Machine contains:
\begin{enumerate}
\item The private key of the Registration Machine.
\item Public keys of all Registration Machines, all Collection
Machines, and all Counting M-Boxes.
\item Machine names (URLs) of other Registration Machines and all Collection
Machines.
\item The unmarked ballot.
\item UIDs and associated public keys of all Private Voting
Machines.  This includes de-registered and test Private Voting
Machines, as Registraton Machines do not know which Private Voting
Machines are de-registered or test machines.
\item A log of any transactions the Registration Machine has
been involved in, including  Certificates issued by the Registration Machine.
This log is public, and allows the public to determine which Private
Voting Machines have voted and when they have voted, though here the
Private Voting Machines are identified by their UID and not by the
name/address of their voter.
\end{enumerate}

A Registration Machine receives two kinds of requests from
Private Voting Machines:
\begin{enumerate}

\item
Request for Ballot.  The reply includes the unmarked ballot
and any public keys and names of Registration and
Collection Machines known to the Registration Machine,
and also the public keys of any M-Boxes known to the Registration
Machine.

\item
Request for Certificate.  The request includes the signature
of an encrypted ballot prepared by the requesting Private
Voting Machine.  The reply is a certificate containing
the signature, the Private Voting Machine's UID,
and a signature of the certificate by the Registation Machine.

\end{enumerate}

Requesters are always Private Voting Machines, and all replies
are sent to just the requester.

A Registration Machine may serve requests from more than
one precinct, but the information flow remains specific
to the precinct.  Each precinct should have at least 3
Registration Machines available.

In order to change Registration Machine URLs and public keys
periodically, a Registration Machine may include
in its Request for Ballot response
URLs and public keys that are to be use after a specified date.
Note that Private Voting Machines do not know the date and time,
unless informed of such by a Registration Machine or the voter,
so if a Private Voting Machine has information 
to be used after a specific date, the Private Voting Machine
will have to ask the voter for the current date.


\section{Private Voting Machine Details}

A Private Voting Machine is run by a single voter
who holds the registration card containing the passkey to
the machine's locked box.  Usually a Private Voting Machine
is used by its voter for multiple elections over a period
of several years, after which the Private Voting Machine
is de-registered and a new Private Voting Machine
is given to the voter.

Initially the locked box of a Private Voting Machine contains:
\newcounter{PVM-COUNTER}
\begin{enumerate}
\item The UID of the Private Voting Machine.
\item The public and private key of the Private Voting Machine.
\item Public keys of some Registration Machines.
\item Machine names (URLs) of some Registration Machines.
\setcounter{PVM-COUNTER}{\value{enumi}}
\end{enumerate}

The Private Voting Machine
makes requests of Registration Machines.  In every case
at least 3 Registration Machines should be used and the
results compared or tabulated to avoid single point failures or corruption.

The first thing a Private Voting Machine does, when prompted by
its voter, is make a Request for Ballot and use the replies to
add to its locked box the following information:
\begin{enumerate}
\setcounter{enumi}{\value{PVM-COUNTER}}
\item Public keys of all Registration Machines, all Collection
Machines, and all Counting M-Boxes.
\item Machine names (URLs) of all Registration Machines and all Collection
Machines.
\item The unmarked ballot.
\end{enumerate}
This can be done days or just minutes before the voter votes.

The next step is for the voter to vote by marking the ballot.

Next the ballot is encrypted and certified.  Encryption of the
ballot is described below.  To certify the ballot, a signature
is computed of the encrypted ballot, and the signature is sent
in a Request to Certify message to each of at least three Registration
Machines.  These in turn send back certificates that are appended
to the encrypted ballot to form the submitted ballot.  Each
certificate contains the signature of the encrypted ballot, the
UID of the Private Voting Machine, and a time stamp.  The certificates
are signed by the Registration Machines but are \underline{not} encrypted.
The certificates are copied to the Registration Machine log, and can be
be viewed by the public.

The submitted ballot is now sent by the Private Voting
Machine to at least three Collection Machines.  The submitted
ballot contains:
\begin{enumerate}
\item The encrypted ballot.
\item Certificates received for the encrypted ballot.  These contain the
UID of the Private Voting Machine.
\item A copy of the unencrypted \underline{unmarked} ballot, for
diagnostic purposes.
\item Version number of the Private Voting Machine software, for
diagnostic purposes.
\end{enumerate}


A voter may change her/his mind and revote.  The last ballot
he/she casts counts.  This is rather important in an election,
as on the one hand a voter may receive new information and
want to change the vote, and on the other hand to avoid system
congestion it is convenient to have votes strung out over time.

\section{Ballot Encryption Details}

The ballot is encrypted as follows.

First, the ballot is randomized by a process something like
the following.  A long random number is generated.  The first
part is used as instructions to place the ballot voting bits in
various random places in the second part of the random number
(ballots have very voting few bits).

Second, a random symmetric key S is generated and used to encrypt
the randomized ballot.  Then 10 more random keys K1, ..., K10 are generated
and used in 5-way Shamir Secret Sharing encryption
scheme to encrypt S.  The 5-way
scheme is such that any 5 of the 10 K-keys can decrypt the encrypted S.
Then each of the 10 K-keys is encrypted by the public key
of a different one of the 10 M-boxes.

The encrypted ballot consists of the randomized encrypted ballot proper,
the encripted key S, the 10 encrypted K-Keys, and for
diagnostic purposes, a copy of the unencrypted, unmarked ballot.  The
encrypted ballot can be decrypted
by any 5 M-boxes.  It does \underline{not} contain the voter's ID.

\section{Certificate Details}

For each encrypted ballot, several certificates are generated
by Registration Machines acting independently.  Each certificate
contains:
\begin{enumerate}
\item UID of Private Voting Machine submitting the encrypted ballot.
\item Signature of the encrypted ballot as computed by the
      Private Voting Machine.  The encrypted ballot itself is
      \underline{never} sent to the Registration Machine.
\item Date and time, as computed by the Registration Machine.
\item Name of the Registration Machine.
\item Signature of the certificate computed using the private key of
the Registration Machine.
\end{enumerate}

\section{Collection Machine Details}

The Collection Machines receive the submitted ballots, each containing
an encrypted ballot and at least 3 certificates for the encrypted
ballot.

At the end of voting the Collection Machines communicate with each
other so all have the same set of submitted ballots.  Then each
Collection Machine validates the ballots and submits all the
valid encrypted ballots, but \underline{without} their certificates, to
each Counting Machine.

A ballot is valid unless:

\begin{enumerate}
\item It has been superceded by a later ballot from the same
Private Voting Machine.
\item It is from a Private Voting Machine that has been de-registered
\pagref{DE-REGISTERED}.
\item It is from a test Private Voting Machine that is
being used as a system/hardware tester \pagref{TESTER}.
\end{enumerate}

Note that messages received that cannot be decrypted using the
private key of the Collection Machine and a public key known
to the Collection Machine are discarded, and so cannot contain
ballots that are either valid or invalid.  Collection Machine
public/private key pairs a election specific, so ballots for
the wrong election will not be considered.

\section{Counting Machine Details}

Counting Machines recieve a complete set of encrypted ballots
from each collection machine, and check to be sure these sets
are the same for each Counting Machine.

Each Counting Machine is then given 5 M-boxes and proceeds to
decrypt and count the ballots.

\section{De-Registration and Testing}

If a Private Voting Machine (i.e., its thumb drive and registration
card) is lost or stolen, the voter may have the machine
de-registered\label{DE-REGISTERED}
and replaced by a new Private Voting Machine.

Some Private Voting Machines are used as `testers'\label{TESTER} for
system software and hardware.
These are used by election officials to cast known votes that can
be checked to ensure that the system and hardware is working as
desired.

To check test votes, the encrypted ballots are submitted to a Test
Machine not otherwise mentioned which is given 5 M-boxes and
decrypts the ballots so they can be compared with hand written
notes taken by the election officials that cast the ballots.
The system does not know which Private Voting Machines are test
machines until after voting ends and the Collection Machines
are given this information.

The hardware used to run test PVMs should be of kinds randomly
choosen from the kinds of hardware used by voters, so that if a
hardware manufacturer is producing hacked hardware that corrupts
ballots, this will be eventually detected.

\section{Vote Buying}

No voting system prevents a person to knows a voter well from
`buying' the voter's vote with some favor.

However buying votes on a large enough scale to influence elections
is difficult to do without being detected and prosecuted.
This is just as true of our proposed E-Election System as of
current voting systems.

The only thing easier to do in the E-Election System is verification
that a vote bought from a stranger
is what the buyer wants it to be.  The E-Election
System shares this with absentee voting systems.  However, since its
nearly impossible to buy large numbers of votes without being detected,
such easier verification is not significant.

\section{Voter Suppression}

\section{Deviant Software}
\label{DEVIANT-SOFTWARE}

\end{document}
